
\newcommand{\vect}[1]{\mathbf{#1}}
\newcommand{\mat}[1]{\mathbf{#1}}
\newcommand{\comp}[1]{#1^{\alpha\beta}}
\newcommand{\norm}[1]{\left|\left|#1\right|\right|}

\begin{document}

\title{ROBUST 3D GRAVITY GRADIENT INVERSION BY\\ PLANTING ANOMALOUS DENSITIES}

% manuscript number
\ms{GEO-2011-0388}

\address{
    \footnotemark[1]
    Observat\'orio Nacional, Geophysics Department\\
    Rio de Janeiro, Brazil\\
    e-mail: leouieda@gmail.com; valcris@on.br}

\author{Leonardo Uieda\footnotemark[1] and
        Val\'eria C. F. Barbosa\footnotemark[1]}

\lefthead{Uieda \& Barbosa}
\righthead{Robust 3D gravity gradient inversion}


{\center
This technical paper was presented in 2011 at 81st SEG Annual Meeting.\\
The original paper was submitted on October 3, 2011.\\[2cm]}

%%%%%%%%%%%%%%%%%%%%%%%%%%%%%%%%%%%%%%%%%%%%%%%%%%%%%%%%%%%%%%%%%%%%%%%%%%%%%%%%
\begin{abstract}
We have developed a new gravity gradient inversion method for
estimating a 3D density-contrast distribution defined on a grid of rectangular
prisms.
Our method consists of an iterative algorithm that does not require the
solution of an equation system. Instead, the solution grows systematically
around user-specified prismatic elements, called ``seeds'', with given density
contrasts.
Each seed can be assigned a different density-contrast value,
allowing the interpretation of multiple sources with different density contrasts
and that produce interfering signals.
In real world scenarios, some sources might not be targeted for the
interpretation. Thus, we developed a robust procedure that neither requires the
isolation of the signal of the targeted sources
prior to the inversion nor requires substantial prior information about
the non-targeted sources.
In our iterative algorithm, the estimated sources grow
by the accretion of prisms in the periphery of the current estimate.
In addition, only the columns of the sensitivity matrix
corresponding to the prisms in the periphery of the current estimate are
needed for the computations.
Therefore, the individual columns of the sensitivity matrix
can be calculated on demand and deleted after an accretion takes place, greatly
reducing the demand for computer memory and processing time.
Tests on synthetic data show the ability of our method to correctly recover the
geometry of the targeted sources, even when interfering
signals produced by non-targeted sources are
present.
Inverting the data from an airborne gravity gradiometry survey flown over the
iron ore province of Quadril\'atero Ferr\'ifero, southeastern Brazil, we
estimate a compact iron ore body that is in agreement with geologic information
and previous interpretations.
\end{abstract}

%%%%%%%%%%%%%%%%%%%%%%%%%%%%%%%%%%%%%%%%%%%%%%%%%%%%%%%%%%%%%%%%%%%%%%%%%%%%%%%%
\section{Introduction}

Historically, the vertical component of the gravity field has been widely
used in exploration geophysics due to the simplicity of its measurement and
interpretation.
This fact propelled the development of a large variety of gravity
inversion methods.
Conversely, the technological difficulties in the acquisition of
accurate airborne gravity gradiometry data resulted in a delay in the
development of methods for the inversion of this kind of data.
Consequently, before the early 1990s, few papers published in the literature
were devoted to the interpretation (or analysis) of gravity gradiometer data.
At this point, two papers deserve the general readers' attention.
The first one is \citet{vasco} which presents a comparative study of the
vertical
component of gravity and the gravity gradient tensor by analyzing their
parameter resolution and variance matrices.
The second paper is \citet{pedersen} which studied data of gravity
and magnetic gradient tensors and introduced scalar invariants that indicate the
dimensionality of the sources.
\\ \indent 
Recent technological developments of moving-platform gravity gradiometers made
it feasible to accurately measure the five linearly independent components
of the gravity gradient tensor.
These technological advances, paired with the advent of global positioning
systems (GPS), have opened a new era in the acquisition of accurate airborne
gravity gradiometry data.
Thus, airborne gravity gradiometry has come to be a useful tool for interpreting
geologic bodies present in both mining and hydrocarbon exploration areas.
Gravity gradiometry has the advantage, compared with other gravity methods,
of being extremely sensitive to localize density contrasts within regional
geological settings \citep{zhdanov10b}.
\\ \indent 
Recently, some gravity gradient inversion algorithms have been adapted to
predominantly interpret both orebodies that are important mineral
exploration targets \citep[e.g.,][]{li, zhdanov04, martinez, wilson},
and salt bodies in a sedimentary setting \citep[e.g.,][]{jorgensen, routh}.
All these methods discretize the Earth's subsurface into prismatic cells with
homogeneous density contrasts and estimate a 3D density-contrast distribution,
thus retrieving an image of geologic bodies.
Usually, a gravity gradient data set contains a huge volume of observations of
the five linearly independent tensor components.
These observations are collected every few meters in surveys that may contain
hundreds to thousands of line kilometers.
This massive data set combined with the discretization of the Earth's subsurface
into a fine grid of prisms results in a large-scale 3D inversion with hundreds
of thousands of parameters and tens of thousands of data.
\\ \indent 
The solution of a large-scale 3D inversion requires overcoming two main
obstacles.
The first one is the large amount of computer memory required to
store the matrices used in the computations, particularly the sensitivity
matrix.
The second obstacle is the computational time
required for matrix-vector multiplications
and to solve the resulting linear system.
One approach to overcome these problems is to use the fast Fourier transform for
matrix-vector multiplications by exploiting the translational invariance of the
kernels to reduce the linear operators to Toeplitz block structure
\citep{pilkington, zhdanov04, wilson}.
However, these approaches are unable to deal with data on an irregular grid or
on an uneven surface.
Furthermore, the observations must lie above the surface topography, so these
approaches cannot be applied to borehole data.
Another strategy for the solution of large-scale 3D inversions involves using a
variety of data compression techniques.
\citet{portniaguine02} use a compression technique based on cubic interpolation.
\citet{li_oldenburg03} use a 3D wavelet compression on each row of the
sensitivity matrix.
Most recently, an alternative strategy for the solution of large-scale 3D
inversion has been used under the name of ``moving footprint''
\citep{cox, zhdanov10a, wilson}.
In this approach the full sensitivity matrix is not computed; rather, for each
row, only the few elements that lie within the radius of the footprint size are
calculated.
In other words, the $j$th element of the $i$th row of sensitivity matrix only
needs to be computed if its distance from the $i$th observation is smaller than
a pre-specified footprint size (expressed in kilometers).
The footprint size is a threshold value defined by the user and will depend on
the natural decay of the Green's function for the gravity field.
The smaller the footprint size, the larger the number of null elements in the
rows of the sensitivity matrix; hence, the faster the inversion will be
but also the greater is the loss of accuracy.
The user can then either accept the result or increase the footprint size and
restart the inversion.
This procedure leads to a sparse representation of the sensitivity matrix
allowing the solution of intractable large-scale 3D inversions via the
conjugate gradient technique.
\\ \indent 
Depending on the regularization function used, inversion methods for estimating
a 3D density-contrast distribution
that discretize the Earth's subsurface into prismatic cells can produce either
blurred images \citep[e.g.,][]{li_oldenburg98} or sharp images of the anomalous
sources \citep[e.g.,][]{portniaguine99, zhdanov04, silvadias09, silvadias11}.
Nevertheless, all of the above-mentioned methods require the solution of a
large linear system, which is, as pointed out before, one of the biggest
computational hurdles for large-scale 3D inversions.
Alternatively, there is a class of gravity inversion methods that do not solve
linear systems but instead search the space of possible solutions for an
optimum one.
This class can be further divided into methods that use random search and
those that use systematic search algorithms.
Among the methods that use random search, we draw attention to the two following
methods.
\citet{nagihara} estimate a 3D density-contrast distribution using the
simulated annealing algorithm (SA).
\citet{krahenbuhl} retrieve a salt body subject to density contrast
constraints by developing a hybrid algorithm that combines the genetic
algorithm (GA) with a modified form of SA as well as a local search technique
that is not activated at every generation of the GA.
On the other hand, examples of methods that use a
systematic search are the methods of
\citet{zidarov}, \citet{camacho}, and
\citet{rene}.
\citeauthor{zidarov}'s \citeyearpar{zidarov} bubbling method looks
for a compact source solution (without hollows in its interior) by transforming
a given initial nonnull spatially discrete density-contrast distribution
$\rho$ inside a region $\Re$ into a constant distribution $\rho^\ast \le \rho$
inside a region $\Re^\ast \supset \Re$ by successive redistribution of the
excess of mass of $\rho$ relative to $\rho^\ast$ in outward directions.
Both distributions fit the gravity data. To overcome the difficulty of setting
an initial density-contrast distribution $\rho$ that not only fits the data,
but also satisfies the constraint that $\rho$ be everywhere greater than or
equal to a specified upper bound, \citet{cordell} adapts the bubbling
method starting with point mass estimates obtained via Euler's homogeneity
equation.
Following the class of systematic search methods, \citet{camacho}
estimates a 3D density-contrast distribution using a systematic search to
iteratively ``grow'' the solution, one prismatic element at a time, from a
starting distribution with zero density contrast.
At each iteration a new prismatic element is added to the estimate with a
pre-specified positive or negative density contrast.
This new prismatic element is chosen by systematically searching the set of all
prisms that still have zero density contrast for the one whose incorporation
into the estimate minimizes a goal function composed of the data-misfit function
plus the $\ell_2$-norm of the weighted 3D density-contrast distribution.
Also belonging to the class of systematic search methods is \citet{rene}, which
is able to recover 2D compact bodies (i.e., with no hollows
inside) with sharp contacts by successively incorporating new prisms around
user-specified prisms called ``seeds''.
These seeds have a given set of density contrasts, all of which must have
the same sign.
At the first iteration, the new prism that will be incorporated is chosen by
systematically searching the set of neighboring prisms of the seeds for the one
that minimizes a ``shape-of-anomaly'' function.
From the second iteration on, the search is performed over the set of
available neighboring prisms of the current estimate.
Thus, the solution grows through the addition of prisms to its periphery, in a
manner mimicking the growth of crystals.
\citeauthor{rene}'s \citeyearpar{rene} method
is restricted to interpret density-contrast distributions with a single
sign and its estimated solution can be allowed to grow in any
combination of user-specified directions.
\\ \indent
These inversion methods that do not solve linear systems have been applied to
the vertical component of the gravity field yielding good results.
To our knowledge, such class of methods has not been previously applied to
interpret gravity gradiometry data.
Besides, these methods are unable to deal with the presence of
interfering signals produced by non-targeted
sources that can be interpreted as geological noise.
This is a common scenario encountered in complex geological settings where the
signal of non-targeted sources cannot be
completed removed from the data.
In the literature, few inversion methods have addressed this issue of
interpreting only targeted sources when in the presence of non-targeted sources
in a geologic setting \citep[e.g.,][]{silva_holmann, silva_cutrim, silvadias07}.
The typical approach is to require the interpreter to perform some sort of data
preprocessing in order to remove the signal
produced by the non-targeted geologic sources.
This preprocessing generally involves filtering the observed data based on the
assumed spectral content of the targeted sources.
However, separating the signal of multiple
sources is often impractical, if not impossible.
An effective way to overcome this problem is to devise an inversion method that
simultaneously estimates targeted geologic sources and
reduces the undesired effects
produced by the non-targeted sources by means of a robust data-fitting
procedure.
\citet{silva_holmann} and \citet{silva_cutrim}, for example, minimized,
respectively, the $\ell_1$-norm and the Cauchy-norm of the residuals (the
difference between the observed and predicted data) to take into account the
presence of non-targeted sources.
Both data-fitting procedures are more robust than the typical least-squares
approach of minimizing the $\ell_2$-norm of the residuals, because they allow
the presence of large residual values.
\\ \indent
We present a new gravity gradient inversion for estimating a 3D density-contrast
distribution belonging to the class of methods that do not solve linear systems,
but instead implement a systematic search algorithm.
Like \citet{rene}, we incorporate prior information into the solution using
seeds (i.e., user-specified prismatic elements) around which the solution grows.
In contrast with \citeauthor{rene}'s \citeyearpar{rene} method, our approach can
be used to interpret multiple geologic sources with density contrasts of
different signs. This is possible because our approach allows
assigning a different density contrast to each seed and does not impose
any restrictions on the sign of the gravity anomaly.
We impose compactness on the solution using a modified version of the
regularizing function proposed by \citet{silvadias09}.
We use as a data-misfit function the
$\ell_1$-norm of the residuals because
it tolerates large data residuals.
This is a desirable feature of the $\ell_1$-norm because it means that it
is less influenced by both outliers in the observed data and non-targeted
sources.
Therefore, our approach requires neither substantial amounts of prior
information about the
non-targeted sources nor the isolation
of the effect of the targeted sources through preprocessing.
Finally, we exploit the fact that our systematic search is limited to the
neighboring prisms of the current estimate to implement a lazy evaluation
\citep{lazyeval} of the
sensitivity matrix, thus achieving a fast and memory efficient inversion.
Tests on synthetic data and on airborne gravity gradiometry data collected over
the Quadril\'atero Ferr\'ifero, southeastern Brazil, confirmed the potential of
our method in producing sharp images of the targeted anomalous density
distribution (iron orebody) in the presence of non-targeted sources.


%%%%%%%%%%%%%%%%%%%%%%%%%%%%%%%%%%%%%%%%%%%%%%%%%%%%%%%%%%%%%%%%%%%%%%%%%%%%%%%%
\section{Methodology}


Let $\vect{g}^{\alpha\beta}$ be an $L$-dimensional vector that contains observed
values of the $g_{\alpha\beta}$ component of the gravity gradient tensor, where
$\alpha$ and $\beta$ belong to the set of $x$-, $y$-, and $z$-directions of a
right-handed Cartesian coordinate system (Figure \ref{fig:interpmodel}).
We define this coordinate system with its
$x$-axis pointing north, $y$-axis pointing east, and $z$-axis pointing down.
We assume that $\vect{g}^{\alpha\beta}$ is caused by an anomalous density
distribution contained within a three-dimensional region of the subsurface.
This region can be discretized into juxtaposed 3D right rectangular prisms
composing an interpretative model (Figure \ref{fig:interpmodel}).
Each prism in this model has a homogeneous density contrast and the resulting
piecewise-constant anomalous density distribution is assumed to be sufficient to
approximate the true one.
It follows that the $g_{\alpha\beta}$ component of the gravity gradient
tensor produced by the anomalous density distribution can be approximated by
the sum of the contributions of each prism of the interpretative model, i.e.,

\begin{equation}
\vect{d}^{\alpha\beta} = \sum\limits_{j=1}^{M} p_j \vect{a}_j^{\alpha\beta}.
\label{eq:dalphabeta_sum}
\end{equation}

\noindent
This linear relationship can be written in matrix notation as

\begin{equation}
\vect{d}^{\alpha\beta} = \mat{A}^{\alpha\beta} \vect{p},
\label{eq:dalphabeta_mat}
\end{equation}

\noindent
where $\vect{p}$ is an $M$-dimensional vector whose
$j$th element, $p_j$, is the density contrast of the $j$th prism of the
interpretative model, $\vect{d}^{\alpha\beta}$ is an
$L$-dimensional vector of data predicted by $\vect{p}$
which one would expect approximates $\vect{g}^{\alpha\beta}$, and 
$\mat{A}^{\alpha\beta}$ is the $L \times M$
sensitivity matrix, whose $j$th column is the
$L$-dimensional vector $\vect{a}_j^{\alpha\beta}$. 
The $i$th element of $\vect{a}_j^{\alpha\beta}$ is numerically equal to the
$g_{\alpha\beta}$ component of the gravity gradient tensor caused by the $j$th
prism of the interpretative model, with unit density contrast, calculated at
the place where the $i$th observation was made.
It is then evident that the $j$th column of the sensitivity
matrix represents the influence that $p_j$ has on the predicted data. 
The elements of matrix $\mat{A}^{\alpha\beta}$ can be calculated using the
formulas of \citet{nagy}.
\\ \indent
Let $\vect{r}^{\alpha\beta}$ be the $L$-dimensional residual
vector of the $g_{\alpha\beta}$ component of the gravity gradient tensor, i.e.,

\begin{equation}
    \vect{r}^{\alpha\beta} = \vect{g}^{\alpha\beta} - \vect{d}^{\alpha\beta}.
    \label{eq:residual}
\end{equation}

\noindent We define the data-misfit function $\phi_{\alpha\beta}(\vect{p})$
of the $g_{\alpha\beta}$ component of the gravity gradient tensor as a norm of
the residual vector $\vect{r}^{\alpha\beta}$. For a least-squares fit,
$\phi_{\alpha\beta}(\vect{p})$ is defined as the
$\ell_2$-norm of the residual vector.
The least-squares fit distributes the residuals assuming that the errors in the
data follow a short-tailed Gaussian distribution and thus sporadically large
residual values are highly improbable \citep{claerbout, silva_holmann, menke,
tarantola}.
Hence, the $\ell_2$-norm is known to be sensitive to outliers in the data, which
can result from either gross errors or geological noise (i.e., anomalous
densities which are not of interest to the interpretation).
On the other hand, if occasional large residuals are desired in the inversion,
one can use the $\ell_1$-norm of the residual vector.
Here, we have chosen the normalized $\ell_1$-norm of the $L$-dimensional
residual vector $\vect{r}^{\alpha\beta}$, hence,
the data-misfit function is defined as

\begin{equation}
\phi_{\alpha\beta}(\vect{p}) =
\frac{\norm{\vect{r}^{\alpha\beta}}_1}{\norm{\vect{g}^{\alpha\beta}}_1} =
\frac{\sum\limits_{i=1}^{L}\left|g_i^{\alpha\beta} - d_i^{\alpha\beta}\right|}
     {\sum\limits_{i=1}^{L}\left|g_i^{\alpha\beta}\right|}.
\label{eq:phi}
\end{equation}

\noindent
In this case, the errors in the data are assumed to follow a long-tailed
Laplace distribution and a more robust fit is obtained since the predicted data
will be insensitive to outliers.
\\ \indent
Let us assume that there are $N_c$ components of the gravity tensor
available.
Hence, the total data-misfit function $\Phi(\vect{p})$ can be defined as the
sum of the individual data-misfit functions for each of the $N_c$ components,
i.e.,

\begin{equation}
\Phi(\vect{p}) = \sum\limits_{k=1}^{N_c}\phi_k(\vect{p}),
\label{eq:Phi}
\end{equation}

\noindent
where $\phi_k(\vect{p})$ is the $k$th function in the set of
$N_c$ available data-misfit functions. For example, if the available components
are $g_{xx}$, $g_{yy}$, and $g_{zz}$, in that order, then $N_c = 3$ and
$\phi_1(\vect{p}) \equiv \phi_{xx}(\vect{p})$, 
$\phi_2(\vect{p}) \equiv \phi_{yy}(\vect{p})$, and
$\phi_3(\vect{p}) \equiv \phi_{zz}(\vect{p})$, all given by equation
\ref{eq:phi}.

\indent
Regardless of the norm used in the data-misfit function,
the inverse problem of
estimating a three-dimensional density-contrast distribution from
gravity gradiometry data is ill-posed and requires additional constraints to be
transformed into a well-posed problem with a unique and stable solution.
The constraints chosen for our method are:

\begin{description}
\item[Constraint 1.] the solution should be compact (i.e., without
    any hollows inside it). 
\item[Constraint 2.] the excess (or deficiency of) mass contained
    in the solution should be concentrated around user-specified prisms of the
    interpretative model with known density contrasts (referred to as
    ``seeds'').
\item[Constraint 3.] the only density-contrast values allowed are
    zero or the values assigned to the seeds.
\item[Constraint 4.] each element of the solution should have the
    density contrast of the seed closest to it.
\end{description}

\noindent We solve the constrained inverse problem of estimating a
parameter vector $\vect{p}$ subject to these constraints through an iterative
algorithm named ``planting algorithm'', as explained bellow.
At each iteration, this algorithm evaluates the goal function

\begin{equation}
\Gamma(\vect{p}) = \Phi(\vect{p}) + \mu\theta(\vect{p}),
\label{eq:goal}
\end{equation}

\noindent
where $\theta(\vect{p})$ is a regularizing function defined in the parameter
(model) space that imposes physical and/or geological attributes on the
solution.
The scalar $\mu$ is a regularizing parameter that balances the tradeoff between
the total data-misfit
function $\Phi(\vect{p})$ (equation \ref{eq:Phi}) and the regularizing
function $\theta(\vect{p})$.
The regularizing function $\theta(\vect{p})$ is an adaptation of the one used in
\citet{silvadias09}, which in turn is a modified version of the one used by
\citet{guillen_menichetti} and \citet{silva_barbosa06}.
It enforces the compactness of the solution and the concentration of mass around
the seeds (i.e., constraints 1 and 2), being defined as

\begin{equation}
\theta(\vect{p}) = \frac{1}{f} \sum\limits_{j=1}^{M} \frac{p_j}{p_j + \epsilon}
    l_j,
\label{eq:regul}
\end{equation}

\noindent
where $p_j$ is the $j$th element of $\vect{p}$, $l_j$ is the distance between
the center of the $j$th prism and the center of a seed
(see subsection Planting algorithm), $\epsilon$ is a small positive
scalar used to avoid a singularity when
$p_j=0$, and the scalar $f$ is the average extent of the interpretative model,
defined as

\begin{equation}
f = \frac{\Delta x + \Delta y + \Delta z}{3},
\label{eq:f}
\end{equation}

\noindent
where $\Delta x$, $\Delta y$, and $\Delta z$ are
the lengths of the interpretative model
in the $x$-, $y$-, and $z$-directions, respectively
(Figure \ref{fig:interpmodel}).
\\ \indent
In practice, the scalar $\epsilon$ (equation \ref{eq:regul}) is not
necessary because one can simply add
either zero or $l_j$ when evaluating the regularizing function.
Furthermore, the value of the regularizing parameter
$\mu$ should be selected through trial and error. 
A small value of $\mu$ is not able to estimate compact sources, whereas a large
value of $\mu$ produces compact solutions that might not fit the observed data.
To determine an adequate value for $\mu$, we start with a small value,
typically $10^{-5}$.
Then, if needed, the value is raised until the estimated
density-contrast distribution achieves the desired compactness.
\\
\indent
The two remaining constraints (3 and 4) are imposed algorithmically, as
explained bellow.

%%%%%%%%%%%%%%%%%%%%%%%%%%%%%%%%%%%%%%%%%%%%%%%%%%%%%%%%%%%%%%%%%%%%%%%%%%%%%%%%
\subsection{Planting algorithm}

Our systematic search algorithm, named ``planting algorithm'', requires
that a set of $N_S$ seeds and their associated density-contrast values be
specified by the user.
Each seed is a prism of the interpretative model.
We emphasize that the density-contrast values of the seeds do not need to be
the same.
These seeds should be chosen according to prior information about
the targeted anomalous sources, such as those provided by the available
geologic models, well logs, and interpretations using other geophysical data
sets.
The planting algorithm starts with an initial
parameter vector that includes the density-contrast values assigned to the seeds
and has all other elements set to zero (Figure \ref{fig:sketch}a).
Hence, by recalling equations 
\ref{eq:dalphabeta_sum} and \ref{eq:residual}, we
can define the initial residual vector of the $g_{\alpha\beta}$ component
of the gravity gradient tensor as

\begin{equation}
\vect{r}_{(0)}^{\alpha\beta} = \vect{g}^{\alpha\beta} -
\left(\sum\limits_{s=1}^{N_S} \rho_s \vect{a}_{j_s}^{\alpha\beta}\right),
\label{eq:residual_initial}
\end{equation}

\noindent
where $\rho_s$ is the density contrast of the $s$th seed, $j_s$ is the
corresponding index of the $s$th seed in the parameter vector $\vect{p}$, and
$\vect{a}_{j_s}^{\alpha\beta}$ is the $L$-dimensional column vector of the
sensitivity matrix $\mat{A}^{\alpha\beta}$ (equation \ref{eq:dalphabeta_mat})
corresponding to the $s$th seed.
We can then proceed to calculate the initial total data-misfit function
$\Phi_{(0)}$ (equation \ref{eq:Phi}), which depends on
$\vect{r}_{(0)}^{\alpha\beta}$.
\\
\indent
The solution to the constrained inverse problem is then built through an
iterative growth process.
Initially, to each seed is assigned a list of its neighboring prisms
(prisms that share a face with the seed).
An iteration of the growth process consists of attempting to grow,
one at a time, each of the $N_S$ seeds by performing the accretion of
a prism from the seed's list of neighboring prisms.
We define the accretion of a prism as changing its density-contrast value from
zero to the density contrast of the seed undergoing the accretion,
guaranteeing constraint 3. 
Thus, a growth iteration is composed of at most $N_S$ accretions,
one for each seed.
Furthermore, constraint 4 is guaranteed because only prisms from the
list of neighboring prisms of the seed undergoing the accretion are eligible to
be accreted to that seed.
\\
\indent
The choice of a neighboring prism for the accretion to the $s$th
seed follows two criteria:

\begin{enumerate}
    \item The addition of the neighboring prism to the current estimate should
    reduce the total data-misfit function
    $\Phi(\vect{p})$ (equations \ref{eq:Phi}), as compared
    to the previous accretion iteration.
    This ensures that the solution grows in a way that best fits the observed
    data.
    To avoid an exaggerated growth of the estimated anomalous densities, the
    algorithm does not perform the accretion of neighboring prisms that produce
    very small changes in the total data-misfit function.
    The criterion for how small a change is accepted is based on whether the
    following inequality holds:

    \begin{equation}
    \frac{|\Phi_{(new)} - \Phi_{(old)}|}{\Phi_{(old)}} \ge \delta,
    \label{eq:delta}
    \end{equation}

    where $\Phi_{(new)}$ is the total data-misfit function
    evaluated with the chosen neighboring prism included in the estimate,
    $\Phi_{(old)}$ is the total data-misfit
    function evaluated during the previous accretion iteration, and $\delta$ is
    a positive scalar typically ranging from $10^{-3}$ to $10^{-6}$.
    Parameter $\delta$ controls how much the anomalous densities are allowed to
    grow.
    The choice of the value of $\delta$ depends on the size of the prisms of the
    interpretative model.
    The smaller the prisms are, the smaller their contribution to
    $\Phi(\vect{p})$ will be and thus the smaller $\delta$ should be.

    \item The addition of the neighboring prism with density contrast $\rho_s$
    to the current estimate should produce the smallest value of the goal
    function $\Gamma(\vect{p})$ (equation \ref{eq:goal}) out of
    all other
    prisms in the list of neighboring prisms of the
    $s$th seed that obeyed the first criterion.
    Thus, the accretion of the neighboring prism to the current estimate will
    produce the highest decrease in the total data-misfit function
    (equation \ref{eq:Phi}) as well as
    the lowest increase in the regularizing function $\theta(\vect{p})$
    (equation \ref{eq:regul}).
    This ensures that constraints 1, 2, and 4 are met.
    We clarify here that the term $l_j$ in equation \ref{eq:regul} is
    the distance between the
    center of the $j$th prism and the center of the $s$th seed (i.e., the
    one that is undergoing the accretion).
    We stress that the $j$th prism belongs to the list of neighboring
    prisms of the $s$th seed.
\end{enumerate}

\indent
Once the accretion of the $j$th
prism is performed to the $s$th seed, the neighboring prisms of the
$j$th prism are included in the the $s$th seed's list of neighboring prisms
and the $j$th prism is removed from this list (Figure
\ref{fig:sketch}b).
It is important to note that the list of seeds is not modified along the
iterations of our algorithm. Rather, the list of neighboring prisms of a seed
varies each time it suffers an accretion.
Finally, we update the residual vectors of the $N_c$
available components. The updated residual vector of the
$g_{\alpha\beta}$ component of the gravity gradient tensor is given by

\begin{equation}
\vect{r}_{(new)}^{\alpha\beta} =
\vect{r}_{(old)}^{\alpha\beta} - p_j\vect{a}_j^{\alpha\beta},
\label{eq:residual_update}
\end{equation}

\noindent
where $\vect{r}_{(new)}^{\alpha\beta}$ is the
updated residual vector,
$\vect{r}_{(old)}^{\alpha\beta}$ is the
residual vector evaluated in the previous accretion iteration, $j$ is the index
of the neighboring prism chosen for the accretion, $p_j = \rho_s$, and
$\vect{a}_j^{\alpha\beta}$ is the $j$th column vector
of the sensitivity matrix $\mat{A}^{\alpha\beta}$.
In the case that none of the neighboring prisms of the $s$th seed meet the first
criterion, the $s$th seed does not grow during this growth iteration. 
This ensures that different seeds can produce anomalous densities of different
sizes.
The growth process continues while at least one of the seeds is able to grow.
At the end of the growth process, our planting algorithm should yield a solution
composed of compact anomalous densities with variable sizes
(Figure \ref{fig:sketch}c).
\citet{Uieda2012} show an animation of this growth process
when the planting algorithm is applied to
synthetic data of the $g_{zz}$ component of the gravity gradient tensor.


%%%%%%%%%%%%%%%%%%%%%%%%%%%%%%%%%%%%%%%%%%%%%%%%%%%%%%%%%%%%%%%%%%%%%%%%%%%%%%%%
\subsection{Lazy evaluation of the sensitivity matrix}

In our planting algorithm all elements of the parameter vector not corresponding
to the seeds start with zero density contrast.
It is then noticeable from equations
\ref{eq:dalphabeta_sum} and \ref{eq:residual_initial} that
the columns of the sensitivity
matrices $\mat{A}^{\alpha\beta}$ that do not correspond to
the seeds are not required for the initial computations.
Moreover, the search for the next element of the parameter vector for the
accretion is restricted to the list of neighboring prisms of the seeds.
This means that the $j$th column vectors
$\vect{a}_j^{\alpha\beta}$ of the sensitivity matrices only need to be
calculated once the $j$th prism of the interpretative model becomes eligible
for accretion (i.e., becomes a member of the list of neighboring prisms
of a seed).
In addition, our algorithm updates the residual vectors after each
successful accretion through equation \ref{eq:residual_update}.
Once the $j$th prism is permanently incorporated
into the current solution, the column vectors
$\vect{a}_j^{\alpha\beta}$ are no longer needed.
Thus, the full sensitivity matrices $\mat{A}^{\alpha\beta}$ are not needed at
any single time during the growth process.
Column vectors of $\mat{A}^{\alpha\beta}$ can be
calculated on demand and deleted once they are no longer required
(i.e., after an accretion).
This technique is known in computer science as a ``lazy evaluation''
\citep{lazyeval}.
Since the computation of the full sensitivity matrix is a
time- and memory-consuming process, the implementation of a lazy evaluation of
$\mat{A}^{\alpha\beta}$
leads to fast inversion times and low memory usage, making viable the inversion
of large data sets using fine grids of prisms for the interpretative models
without needing supercomputers or data compression algorithms
\citep[e.g.,][]{portniaguine02}. 


%%%%%%%%%%%%%%%%%%%%%%%%%%%%%%%%%%%%%%%%%%%%%%%%%%%%%%%%%%%%%%%%%%%%%%%%%%%%%%%%
\subsection{Presence of non-targeted sources}

In real world scenarios there are interfering signals
produced by multiple and horizontally separated sources (Figure
\ref{fig:robust}a).
Some of these sources may be of no interest to the interpretation
(i.e., non-targeted sources) or there may not be enough available prior
information about them, like their approximate depths or density contrasts.
Furthermore, in most cases it is not possible to separate the
signal of the targeted and the non-targeted sources.
It would then be desirable to provide seeds only for the targeted sources and
that the estimated density-contrast distribution could be obtained without
being affected by the signal of the non-targeted sources.
For this purpose, one can use the $\ell_1$-norm of the residual vector
(equation \ref{eq:phi})
to allow large residual values in the signal that is most influenced
by the non-targeted sources (Figure \ref{fig:robust}b).
Thus, the inversion is less influenced by the signal
yielded by the non-targeted sources by treating it as outliers in the data.
Note that the $\ell_1$-norm by itself does not ``know'' which parts of the
data should be treated as outliers.
This information is indirectly incorporated into the inversion through the
locations of the seeds provided for the targeted sources only.
Therefore, the $\ell_1$-norm has to be used in conjunction with the strong
mass-concentration constraints imposed by the regularizing function
(equation \ref{eq:regul}).
\\ \indent
This robust procedure allows one to choose the targets of the interpretation
without having to isolate their signal before performing the inversion.
It also eliminates the need for prior information about the
density contrast and approximate depth of the
non-targeted sources, although their approximate horizontal locations
are still required.
Additionally, by using the $\ell_1$-norm of the residual vector we
can also handle noisy outliers in the data, such as instrumental or operational
errors.


%%%%%%%%%%%%%%%%%%%%%%%%%%%%%%%%%%%%%%%%%%%%%%%%%%%%%%%%%%%%%%%%%%%%%%%%%%%%%%%%
\section{Application to synthetic data}

We applied our method to synthetic
noise-corrupted data of the $g_{yy}$, $g_{yz}$, and $g_{zz}$ components
of the gravity gradient tensor.
Figure \ref{fig:synthdata}a shows a color-scaled map of the synthetic
$g_{zz}$ component.
Color-scaled maps of the $g_{yy}$ and $g_{yz}$ components
are provided in Figure 1 of
the supplementary material of \citet{Uieda2012b}.
The synthetic data were produced by 11 rectangular parallelepipeds (Figure
\ref{fig:synthresult}a) with density contrasts ranging from
$-1\ \mathrm{g/cm}^3$ to $1.2\ \mathrm{g/cm}^3$.
Each component was calculated on a regular grid of $51 \times 51$ observation
points in the $x$- and $y$-directions, totaling 7,803 observations, with a grid
spacing of 0.1 km along both directions.
We corrupted the synthetic data with pseudorandom Gaussian noise with zero mean
and 5 E\"otv\"os standard deviation.
\\ \indent
To demonstrate the efficiency of our method in retrieving only the targeted
sources even in the presence of non-targeted ones, we chose only the sources
with density contrast of $1.2\ \mathrm{g/cm}^3$ (red blocks in Figure
\ref{fig:synthresult}a) as targets of the interpretation.
Thus, we specified the set of 13 seeds shown in Figure
\ref{fig:synthresult}b (nine for the largest
source and four for the smallest one) and used the $\ell_1$-norm of the residual
vector (equation \ref{eq:phi}) to ignore the signal of the non-targeted
sources (all sources with density contrast different from
$1.2\ \mathrm{g/cm}^3$) displayed as blue and yellow blocks in Figure
\ref{fig:synthresult}a.
The inversion was performed using an interpretative model consisting of 37,500
juxtaposed rectangular prisms, $\mu=10^{-1}$, and $\delta = 10^{-4}$.
We used the $g_{yy}$ and $g_{yz}$ components, as well as $g_{zz}$, because
these two components emphasize
the signal of the targeted sources, which are
elongated in the $x$-direction.
\\ \indent
Figure \ref{fig:synthdata}a shows the predicted data
(black contour lines)
of the $g_{zz}$ component
produced by the estimated density-contrast distribution
shown in Figure \ref{fig:synthresult}c.
The predicted data of the $g_{yy}$ and $g_{yz}$ components
are provided in Figure 1 of
the supplementary material of \citet{Uieda2012b}.
By comparing the density-contrast estimates (Figure
\ref{fig:synthresult}c) with the true targeted
sources (red blocks in Figure \ref{fig:synthresult}a),
we verify the good performance of our method
in recovering targeted sources in the presence of non-targeted sources
(blue and yellow blocks in Figure \ref{fig:synthresult}a)
yielding interfering signals.
The most striking feature of this inversion result is that it required neither
prior information about the density contrasts and approximate depths of the
non-targeted sources nor a signal separation to isolate the effect of the
targeted sources.
For comparison, Figure \ref{fig:synthdata}b shows
a colored-contour map of the $g_{zz}$ component of the gravity gradient tensor
produced by the targeted sources only
(red blocks in Figure \ref{fig:synthresult}a)
plotted against the predicted data (black contour lines in Figure
\ref{fig:synthdata}a-b) produced by the estimated
density-contrast distribution (Figure \ref{fig:synthresult}c).
Notice that the inversion performed on the synthetic data set produced by both
targeted and non-targeted sources (color-scale map in Figure
\ref{fig:synthdata}a) was able to fit the isolated signals
produced by the targeted sources as shown in Figure
\ref{fig:synthdata}b (black contour lines).
These results confirm the ability of our method to
tolerate the large residuals caused by
the non-targeted sources and successfully recover the
targets of the interpretation.
Furthermore, when performed on a standard laptop computer with an
Intel\textregistered ~ Core\texttrademark ~ 2 Duo P7350 2.0 GHz processor, the
total time for the inversion was approximately 46 seconds.
\\ \indent
In \citet{Uieda2012b},
we also present a synthetic example
illustrating the use of
the normalized $\ell_{2}$-norm of the residual vector
(equation \ref{eq:residual})
in the data-mist function in
a geologic setting composed of targeted sources only.

%%%%%%%%%%%%%%%%%%%%%%%%%%%%%%%%%%%%%%%%%%%%%%%%%%%%%%%%%%%%%%%%%%%%%%%%%%%%%%%%
\section{Sensitivity analysis}

We present two analyses of important characteristics of our method.
In the first one, we investigate the sensitivity of our method to
uncertainties in the a priori information
(i.e., location and density contrasts of the seeds).
In the second analysis, we investigate the limitations of the robust procedure
that was
proposed to deal with the presence of non-targeted sources.
For these purposes, we have conducted various tests on synthetic noise-corrupted
data produced by two contiguous sources: a larger dipping source with density
contrast of $1\ \mathrm{g/cm}^3$, and a smaller cubic source with density
contrast of $-1\ \mathrm{g/cm}^3$
(black outline in Figures \ref{fig:fullseeds},
\ref{fig:wrongseeds},
\ref{fig:oneseed}, and
\ref{fig:samesign}).
The depth to the tops of both sources is 0.2 km.
All tests were undertaken on the $g_{xx}$, $g_{xy}$, $g_{xz}$, $g_{yy}$,
$g_{yz}$, and $g_{zz}$ components of the gravity gradient tensor,
which were computed at 150 meter height on a
regular grid of $31 \times 31$ observation points and with grid spacing of
1 km along the $x$- and $y$-directions.
The data were contaminated with pseudorandom Gaussian noise with zero mean and
standard deviation of 0.5 E\"otv\"os.
The interpretative model used in the inversions consists of 27,000 juxtaposed
right rectangular prisms.
In all tests, only the large dipping source was the target of the
interpretation.
\\ \indent
In the first test, we assigned three seeds
(gray prisms in the inset of Figure \ref{fig:fullseeds})
with density contrast of $1\ \mathrm{g/cm}^3$ to the targeted dipping source.
These seeds have ideal locations and correctly describe the true framework of
the targeted source.
Figure \ref{fig:fullseeds} shows
the estimated density-contrast distribution
obtained by setting the inversion control variables
$\mu = 1$ and $\delta = 10^{-4}$.
This result demonstrates the excellent performance of our method in recovering
the true target dipping source in the presence of the non-targeted cubic source
with density contrast of $-1\ \mathrm{g/cm}^3$.
The standard deviation of the residual vector of the $g_{zz}$ component
(equation \ref{eq:residual})
is 0.54 E\"otv\"os,
which shows that the predicted data fit the synthetic data
within the data error level of 0.5 E\"otv\"os
that was used to contaminate the data.
This test represents an ideal scenario and
will be used as a baseline
for comparison with subsequent tests.
\\ \indent
The second test was designed to assess the sensitivity of the planting algorithm
to uncertainties in the density-contrast value of the targeted sources.
Thus, we used the same seed locations and inversion control variables as in the
first test, but assigned density contrasts to the seeds that were smaller and
larger than the true value.
The standard deviation of the residual vector of the $g_{zz}$ component
was 0.53 E\"otv\"os,
for the case with a smaller density contrast,
and 0.56 E\"otv\"os, for the case with a larger density contrast.
Hence, in both cases,
the predicted data fit the synthetic data
within the assumed data error level.
Furthermore, the estimated density-contrast distributions
\citep[see Figures 2 and 3 in the supplementary material of][]{Uieda2012b}
are compact and present the correct dip of the targeted dipping source.
However, by assigning a density contrast smaller than the true one, the
estimated density-contrast distribution
\citep[see Figure 2 in][]{Uieda2012b}
displays a larger volume when compared to the true source.
On the other hand, by assigning a density contrast larger than the true one,
the estimated density-contrast distribution
\citep[see Figure 3 in][]{Uieda2012b}
has a smaller volume when compared to the true source.
\\ \indent
The third test had the purpose of assessing the sensitivity of our method
to the wrong positioning of the seeds that define framework of the targeted
source.
For this purpose, we used three seeds
(gray prisms in the inset of Figure \ref{fig:wrongseeds})
with the correct density contrast of $1\ \mathrm{g/cm}^3$
but with their positions defining the wrong dip of the true targeted dipping
source.
We set $\mu=1$ and $\delta=10^{-4}$.
Despite the error in defining the locations of the seeds, the estimated
density-contrast distribution
(Figure \ref{fig:wrongseeds})
still retains the main feature of the true targeted source.
However, the solution is not compact and
the standard deviation of the residual vector of the $g_{zz}$ component
is 0.70 E\"otv\"os,
which shows that the predicted data does not explain
the synthetic data
within the assumed data error level.
\\ \indent
In the fourth test, we assessed the sensitivity of our method to a substantial
reduction in the number of seeds.
Hence, we assigned a single seed
(gray prism in the inset of Figure \ref{fig:oneseed})
with density contrast of $1\ \mathrm{g/cm}^3$.
The choice of positioning the seed at the top of the targeted dipping source is
based on a hypothetical previous interpretation provided, for example, by Euler
deconvolution.
We performed several inversions by setting $\delta=10^{-4}$ and varying
$\mu$ from 1 to $10^{10}$.
The estimated density-contrast distribution
(Figure \ref{fig:oneseed})
is not compact and is not able to
reconstruct the true targeted dipping source, even when $\mu$ is assigned a
large value (e.g., $10^{10}$).
Additionally,
the standard deviation of the residual vector of the $g_{zz}$ component is
2.01 E\"otv\"os, which shows that
the synthetic data are not fitted by the predicted data
within the assumed errors.
\\ \indent
The fifth test was meant to analyze the limitations of the proposed robust
procedure to effectively ignore the interfering signal of the non-targeted
cubic source.
We kept the targeted dipping source as it was and changed the density-contrast
value of the non-targeted cubic source to $1.5\ \mathrm{g/cm}^3$.
This was done in order to simulate targeted and non-targeted sources with
density contrasts of the same sign.
We used the same seeds as in the first test
(inset of Figure \ref{fig:fullseeds})
with the correct density contrast of $1\ \mathrm{g/cm}^3$.
These seeds correctly describe the true framework of the targeted dipping
source.
The inversion was performed using $\mu=1$ and $\delta=10^{-4}$.
We found that the estimated density-contrast distribution
(Figure \ref{fig:samesign})
is not compact and does not retrieve the true targeted source.
However,
the standard deviation of
the residual vector of the $g_{zz}$ component
is 0.71 E\"otv\"os,
which shows that this solution does not explain
the synthetic data within the assumed data error level.
\\ \indent
Figure 4 of the supplementary material of \citet{Uieda2012b}
shows the synthetic noise-corrupted and predicted data
of the $g_{zz}$ component of the gravity gradient tensor
for all tests.


%%%%%%%%%%%%%%%%%%%%%%%%%%%%%%%%%%%%%%%%%%%%%%%%%%%%%%%%%%%%%%%%%%%%%%%%%%%%%%%%
\section{Application to real data}

One of the most important iron provinces in Brazil is the Quadril\'atero
Ferr\'ifero (QF), located in the S\~ao Francisco Craton, southeastern Brazil.
Most of the iron ore bodies in the QF are hosted in the oxided, metamorphosed
and heterogeneously deformed Banded Iron Formation (BIF) of the Cau\^e
Formation, the so-called itabirites.
The itabirites are associated with the Minas Supergroup and contain iron ore
oxide facies, such as hematites, magnetites and martites.
We applied our method to estimate the geometry and extent of the iron ore
deposits of the Cau\^e Formation using the data from an airborne gravity
gradiometry survey performed in this area (color-scale maps in Figure
\ref{fig:realdata}a-c).
The signals associated with the iron ore bodies (targeted sources)
are more prominent in the $g_{yy}$, $g_{yz}$, and $g_{zz}$ components of the
measured gravity gradient tensor (elongated southwest-northeast feature in
Figure \ref{fig:realdata}a-c).
This data set also shows interfering signals caused by
other sources, which will be considered as non-targeted sources in our
interpretation.
\\ \indent
The inversion was performed on 4,582 measurements of each of the $g_{yy}$,
$g_{yz}$, and $g_{zz}$ components of the gravity gradient tensor resulting in a
total of 13,746 measurements.
We applied our robust procedure in order to recover only the targeted sources
(iron ore bodies) in the presence of the non-targeted sources.
Thus, we used the $\ell_1$-norm of the residual vector (equation \ref{eq:phi})
and provided a set of 46 seeds (black stars in Figure
\ref{fig:realdata}) for
the targeted iron ore bodies of the Cau\^e Formation.
The horizontal locations of the seeds were chosen based on the peaks of the
elongated southwest-northeast positive feature (associated with the iron ore
bodies) in the color-scale map of the $g_{zz}$ component (Figure
\ref{fig:realdata}c).
The depths of the seeds were chosen based on borehole information and previous
geologic interpretations of the area.
We assigned a density-contrast value of $1\ \mathrm{g/cm}^3$ for the seeds
because the data were terrain corrected using a density of
$2.67\ \mathrm{g/cm}^3$ and the assumed density of the iron ore deposits is
$3.67\ \mathrm{g/cm}^3$.
The interpretative model was formed by a regular mesh cropped to the area of
interest and consists of 164,892 prisms which follow the topography of the area
(Figure \ref{fig:realresult}a).
The inversion was performed using $\mu=0.1$ and $\delta = 5 \times 10^{-5}$.
\\ \indent
The estimated density-contrast distribution corresponding to
the iron ore bodies of the Cau\^{e} itabirite
is shown in red in Figure \ref{fig:realresult}.
Cross-sections of the estimated density contrast distribution (Figure
\ref{fig:realcross}) show
that the estimated iron ore bodies are compact and have non-outcropping parts.
Figure \ref{fig:realdata}d-f shows the predicted data caused by
the estimated density-contrast distribution shown in
Figure \ref{fig:realresult}.
For all three components, the inversion is able to fit the elongated
southwest-northeast feature associated with the iron ore deposits (targeted
sources) and successfully ignore the other signals presumably produced by the
non-targeted sources (Figure \ref{fig:realdata}).
These results show the ability of our method to provide a compact estimate of
the iron ore deposits.
We emphasize that this was possible without prior information about
the density contrasts and approximate depths of the non-targeted sources
or without isolating the signals produced by the targeted sources.
All these requirements would be impractical in this highly complex geological
setting.
Our results are in close agreement with previous interpretations by
\citet{martinez}.
Furthermore, when performed on a standard laptop computer with an
Intel\textregistered ~ Core\texttrademark ~ 2 Duo P7350 2.0 GHz processor, the
total time for the inversion was approximately 14 minutes.

%%%%%%%%%%%%%%%%%%%%%%%%%%%%%%%%%%%%%%%%%%%%%%%%%%%%%%%%%%%%%%%%%%%%%%%%%%%%%%%%
\section{Discussion}

The proposed inversion method
incorporates a priori information into the
solution through user-specified seeds.
The positions of the seeds determine roughly where the ``skeletons'' of the
estimated targeted sources will be.
Whereas, the density-contrast values assigned to the seeds determine the density
contrasts of the estimated targeted sources.
Therefore, one must provide adequate seeds in order to obtain good results.
In cases where the density-contrast values of the seeds are poorly assigned,
the volumes of the estimated sources will be either greater or smaller than
the true ones.
However, their overall shape and mass do not appear to be affected.
Tests on synthetic data
(Figures \ref{fig:wrongseeds} and \ref{fig:oneseed})
indicate that a reasonable fit of the observed data is not
obtained if the number of seeds used or their positions are inadequate.
Moreover, in these cases our method is not able to estimate compact sources.
Rather, the estimated sources exhibit shapes that do not resemble geologic
structures, such as the tentacle-like structures shown in
Figures \ref{fig:wrongseeds} and \ref{fig:oneseed}.
Thus, the presence of these ``tentacles'' in a solution, combined with a poor
fit of the observed data may be used as heuristic criteria to evaluate the
correctness of the locations of the seeds.
In cases where the errors in the locations are small, the direction in which
the tentacles grow may indicate the direction in which lies a better
position for the seeds (Figure \ref{fig:wrongseeds}).
Thus, the positions of the seeds can be manually adjusted by the user
until an acceptable data fit is obtained and the estimated sources are
not only compact, but resemble geologic structures.
We emphasize that this procedure is only practical because our method is
computationally efficient, which is due to the restricted systematic search of
the planting algorithm and the lazy evaluation of the sensitivity matrix.
An alternative approach to determine the locations of the seeds is to use
interpretation methods that estimate the centers of mass of the sources
\citep[e.g.,][]{medeiros, beiki}.
\citet{medeiros} achieve this by inverting the source moments obtained from the
gravity anomaly.
On the other hand, \citet{beiki} use the eigenvectors of the gravity gradient
tensor to estimate the coordinates of the centers of mass of the sources.
However, we stress that if the fit of the observed data is acceptable and the
estimated sources present geologically reasonable shapes, the hypothesis about
the seeds must be accepted.
Hence, the estimated solution must be accepted as a possible solution, even if
it differs from the true one.
\\ \indent
Another type of a priori information required by our method is whether or not
a given signal is due to the targeted sources.
This information is conveyed through the horizontal locations of the seeds
associated with the targeted sources only.
Because of this information and the mass concentration constraint imposed by the
regularizing function (equation \ref{eq:regul}), the estimated sources cannot
grow too far from the seeds.
Furthermore, the use of the $\ell_1$-norm of the residual vectors (equation
\ref{eq:phi}) reduces the influence of the signal of the non-targeted sources.
Nonetheless, the use of the $\ell_1$-norm alone cannot guaranteed the
robustness of our method to the presence of non-targeted sources.
Tests on synthetic data
(Figures \ref{fig:synthresult} and \ref{fig:samesign})
show that the robustness requires some form of ``barrier''
between the targeted and non-targeted sources.
We concluded from our tests that these barriers must be sources with a density
contrast of opposite sign of the targeted sources.
These barriers work as a natural obstacle for the growth of the estimated
density-contrast distribution (Figure \ref{fig:synthresult}).
We also stress that even in the case where the targeted and non-targeted sources
have opposite signs, the robustness of our method may fail if the signals
produced by these sources present a substantial overlap.
In this case the estimated volume of the targeted source will be
underestimated.

%%%%%%%%%%%%%%%%%%%%%%%%%%%%%%%%%%%%%%%%%%%%%%%%%%%%%%%%%%%%%%%%%%%%%%%%%%%%%%%%
\section{Conclusions}

We have presented a new method for the 3D inversion of gravity gradient data
that uses a systematic search algorithm.
We parametrized the Earth's subsurface as a grid of juxtaposed right rectangular
prisms with homogeneous density contrasts.
The estimated density-contrast distribution is then iteratively built through
the successive accretion of new elements around user-specified prisms called
``seeds''.
The choice of seeds is used to incorporate into the solution prior information
about the density-contrast values and the approximate location of the sources.
Our method is able to retrieve multiple sources with different locations,
geometries, and density contrasts by allowing each seed to have a different
density contrast.
Furthermore, we devised a robust procedure that, in some situations, recovers
only targeted sources when in the presence of non-targeted sources that yield
interfering signals.
Thus, prior information about density contrasts and approximate depths of
the non-targeted sources is not required.
In addition, the signal of the targeted sources does not need to
be previously isolated in order to perform to the inversion.
In real world scenarios, both of the previously stated requirements would be
highly impractical, or even impossible.
\\ \indent
The developed inversion method has low processing time and computer memory
usage since there are no matrix multiplications or linear systems to be solved.
Further computational efficiency is achieved by implementing a
``lazy evaluation'' of the sensitivity matrix.
These optimizations make feasible the inversion of the large data sets
encountered in airborne gravity gradiometry surveys while using an
interpretative model composed of a large number of prisms.
Tests on synthetic data and real data from an airborne gravity gradiometry
survey show that our method is able to recover compact bodies despite the
presence of interfering signals produced by non-targeted sources.
However, the developed method requires a substantial amount of a priori
information.
Thus, it is not suitable for interpretations on a regional scale lacking
detailed geologic information.
Instead, our method should be applied on localized high-resolution
interpretations of well constrained targets.
This makes our inversion method more suitable to be employed in later stages of
an exploration program, when geological mappings and boreholes are available.
Ergo, ideal geologic targets would be compact three-dimensional bodies
with sharp boundaries, like salt domes, orebodies, and igneous intrusions.


%%%%%%%%%%%%%%%%%%%%%%%%%%%%%%%%%%%%%%%%%%%%%%%%%%%%%%%%%%%%%%%%%%%%%%%%%%%%%%%%
\newpage
\section{Acknowledgments}

The authors thank assistant editor Jose Carcione,
associate editor Xiong Li,
reviewer Gary Barnes, and three anonymous reviewers for their questions
and suggestions that greatly improved the original manuscript.
We thank Vanderlei C. Oliveira Junior,
Dionisio U. Carlos,
Irineu Figueiredo,
Eder C. Molina, and
Jo\~ao B. C. Silva for discussions and insightful comments.
We acknowledge the use of plotting library matplotlib by \citet{matplotlib} and
software Mayavi by \citet{mayavi}.
The authors were supported in this research by
a fellowship (VCFB) from
Conselho Nacional de Desenvolvimento Cient\'ifico e Tecnol\'ogico (CNPq) and
a scholarship (LU) from
Coordena\c{c}\~ao de Aperfei\c{c}oamento de Pessoal de N\'ivel Superior (CAPES),
Brazil.
Additional support for the authors was provided by
the Brazilian agencies CNPq (grant 471693/2011-1) and
FAPERJ (grant E-26/103.175/2011).
The authors would like to thank Vale for permission to use the gravity
gradiometry data of the Quadril\'atero Ferr\'ifero.

%%%%%%%%%%%%%%%%%%%%%%%%%%%%%%%%%%%%%%%%%%%%%%%%%%%%%%%%%%%%%%%%%%%%%%%%%%%%%%%%
\newpage
\bibliographystyle{seg}
\begin{thebibliography}{}
\itemsep0pt

\bibitem[Beiki and Pedersen, 2010]{beiki}
Beiki, M., and L. B. Pedersen, 2010, Eigenvector analysis of gravity
    gradient tensor to locate geologic bodies: Geophysics, {\bf 75}, no. 6,
    I37--I49, doi:10.1190/1.3484098.
    
\bibitem[Camacho et al., 2000]{camacho}
Camacho, A. G., F. G. Montesinos, and R. Vieira, 2000, Gravity inversion by
    means of growing bodies: Geophysics, {\bf 65}, 95--101,
    doi:10.1190/1.1444729.

\bibitem[Claerbout and Muir, 1973]{claerbout}
Claerbout, J. F., and F. Muir, 1973, Robust modeling with erratic
    data: Geophysics, {\bf 38}, 826--844, doi: 10.1190/1.1440378.

\bibitem[Cordell, 1994]{cordell}
Cordell, L., 1994, Potential-field sounding using Euler's homogeneity equations
    and Zidarov bubbling: Geophysics, {\bf 59}, 902--908.

\bibitem[Cox et al., 2010]{cox}
Cox, L. H., G. Wilson, and M. S. Zhdanov, 2010, 3D inversion of airborne
    electromagnetic data using a moving footprint: Exploration Geophysics,
    {\bf 41}, 250--259, doi: 10.1071/EG10003.

\bibitem[Guillen and Menichetti, 1984]{guillen_menichetti}
Guillen, A., and V. Menichetti, 1984, Gravity and magnetic inversion with
    minimization of a specific functional: Geophysics, {\bf 49}, 1354--1360,
    doi:10.1190/1.1441761.

\bibitem[Henderson and Morris, 1976]{lazyeval}
Henderson, P., and J. H. Morris, Jr., 1976, A lazy evaluator: Proceedings
    of the 3rd ACM SIGACT-SIGPLAN symposium on Principles on programming
    languages, ACM, 95--103, doi:10.1145/800168.811543.

\bibitem[Hunter, 2007]{matplotlib}
Hunter, J. D., 2007, Matplotlib: A 2D graphics environment: Computing in Science
    and Engineering, {\bf 9}, 90--95, doi:10.1109/MCSE.2007.55.
    
\bibitem[Jorgensen and Kisabeth, 2000]{jorgensen}
Jorgensen, G. J., and J. L. Kisabeth, 2000, Joint 3D inversion of gravity,
    magnetic and tensor gravity fields for imaging salt formations in the
    deepwater Gulf of Mexico: 70th Annual International Meeting, SEG, Expanded
    Abstracts, 424--426.
    
\bibitem[Krahenbuhl and Li, 2009]{krahenbuhl}
Krahenbuhl, R. A., and Y. Li, 2009, Hybrid optimization for lithologic inversion
    and time-lapse monitoring using a binary formulation: Geophysics, {\bf 74},
    no. 6, I55--I65, doi:10.1190/1.3242271.

\bibitem[Li, 2001]{li}
Li, Y., 2001, 3D inversion of gravity gradiometer data: 71st Annual
    International Meeting, SEG, Expanded Abstracts, 1470--1473.

\bibitem[Li and Oldenburg, 1998]{li_oldenburg98}
Li, Y., and D. W. Oldenburg, 1998, 3-D inversion of gravity data: Geophysics,
    {\bf 63}, 109--119, doi: 10.1190/1.1444302.

\bibitem[Li and Oldenburg, 2003]{li_oldenburg03}
\rule{1cm}{.4pt}, 2003, Fast inversion of large-scale magnetic data using
    wavelet transforms and a logarithmic barrier method: Geophysical Journal
    International, {\bf 152}, 251--265, doi: 10.1046/j.1365-246X.2003.01766.x.

\bibitem[Martinez et al., 2010]{martinez}
Martinez, C., Y. Li, R. Krahenbuhl, and M. Braga, 2010, 3D Inversion of airborne
    gravity gradiometry
    for iron ore exploration in Brazil: 80th Annual
    International Meeting, SEG, Expanded Abstracts, 1753--1757. 

\bibitem[Medeiros and Silva, 1995]{medeiros}
Medeiros, W. E., and J. B. C. Silva, 1995, Gravity source moment inversion:
    A versatile approach to characterize position and 3-D orientation of
    anomalous bodies: Geophysics, {\bf 60}, 1342--1353.

\bibitem[Menke, 1989]{menke}
Menke, W., 1989, Geophysical Data Analysis: Discrete Inverse Theory, volume 45
    of International Geophysics Series: Academic Press Inc.

\bibitem[Nagihara and Hall, 2001]{nagihara}
Nagihara, S., and S. A. Hall, 2001, Three-dimensional gravity inversion using
    simulated annealing: Constraints on the diapiric roots of allochthonous salt
    structures: Geophysics, {\bf 66}, 1438--1449, doi:10.1190/1.1487089.

\bibitem[Nagy et al., 2000]{nagy}
Nagy, D., G. Papp, and J. Benedek, 2000, The gravitational potential and its
    derivatives for the prism: Journal of Geodesy, {\bf 74}, 552--560,
    doi: 10.1007/s001900000116. 

\bibitem[Pedersen and Rasmussen, 1990]{pedersen}
Pedersen, L. B., and T. M. Rasmussen, 1990, The gradient tensor of  potential
    field anomalies: Some implications on data collection and data processing of
    maps: Geophysics, {\bf 55}, 1558--1566, doi:10.1190/1.1442807.

\bibitem[Pilkington, 1997]{pilkington}
Pilkington, M., 1997, 3-D magnetic imaging using conjugate gradients:
    Geophysics, {\bf 62}, 1132--1142, doi: 10.1190/1.1444214.

\bibitem[Portniaguine and Zhdanov, 1999]{portniaguine99}
Portniaguine, O., and M. S. Zhdanov,  1999, Focusing geophysical inversion
    images: Geophysics, {\bf 64}, 874--887, doi: 10.1190/1.1444596.

\bibitem[Portniaguine and Zhdanov, 2002]{portniaguine02}
\rule{1cm}{.4pt}, 2002, 3-D magnetic inversion with data compression and image focusing:
    Geophysics, {\bf 67}, 1532--1541, doi: 10.1190/1.1512749.

\bibitem[Ramachandran and Varoquaux, 2011]{mayavi}
Ramachandran, P., and G. Varoquaux, 2011, Mayavi: 3D visualization of scientific
    data: Computing in Science and Engineering, {\bf 13}, 40–50,
    doi:10.1109/MCSE.2011.35
    
\bibitem[Ren\'e, 1986]{rene}
Ren\'e, R. M., 1986, Gravity inversion using open, reject, and
    ``shape-of-anomaly'' fill criteria: Geophysics, {\bf 51}, 988--994,
    doi:10.1190/1.1442157.

\bibitem[Routh et al., 2001]{routh}
Routh, P. S., G. J. Jorgensen, and J. L. Kisabeth, 2001, Base of the salt
    imaging using gravity and tensor gravity data: 71st Annual International
    Meeting, SEG, Expanded Abstracts, 1482--1484.

\bibitem[Silva Dias et al., 2007]{silvadias07}
Silva Dias, F. J. S., V. C. F. Barbosa, and J. B. C. Silva, 2007, 2D gravity
    inversion of a complex interface in the presence of interfering sources:
    Geophysics, {\bf 72}, no. 2, I13--I22, doi: 10.1190/1.2424545.

\bibitem[Silva Dias et al., 2009]{silvadias09}
\rule{1cm}{.4pt}, 2009, 3D gravity inversion through an adaptive-learning procedure:
    Geophysics, {\bf 74}, no. 3, I9--I21, doi:10.1190/1.3092775.

\bibitem[Silva Dias et al., 2011]{silvadias11}
\rule{1cm}{.4pt}, 2011, Adaptive learning 3D gravity inversion for salt-body imaging:
    Geophysics, {\bf 76}, no. 3, I49--I57, doi:10.1190/1.3555078.

\bibitem[Silva and Barbosa, 2006]{silva_barbosa06}
Silva, J. B. C., and V. C. F. Barbosa, 2006, Interactive gravity inversion:
    Geophysics, {\bf 71}, no. 1, J1--J9, doi:10.1190/1.2168010.

\bibitem[Silva and Cutrim, 1989]{silva_cutrim}
Silva, J. B. C., and A. O. Cutrim, 1989, A Robust Maximum Likelihood Method for
    Gravity and Magnetic Interpretation: Geoexploration, {\bf 26}, 1--31,
    doi: 10.1016/0016-7142(89)90017-3. 

\bibitem[Silva and Holmann, 1983]{silva_holmann}
Silva, J. B. C., and G. W. Holmann, 1983, Nonlinear magnetic inversion using a
    random search method: Geophysics, {\bf 48}, 1645--1658.

\bibitem[Tarantola, 2005]{tarantola}
Tarantola, A., 2005, Inverse problem theory and methods for model parameter
    estimation: Society for Industrial and Applied Mathematics.
    
\bibitem[Uieda and Barbosa, 2012a]{Uieda2012}
Uieda, L., and V. C. F. Barbosa, 2012a, {Animation of growth iterations during 3D
    gravity gradient inversion by planting anomalous densities}: Figshare,
    http://hdl.handle.net/10779/2f26602b43f73723987b8d04946bfa41,
    accessed 25 April 2012.
    
\bibitem[Uieda and Barbosa, 2012b]{Uieda2012b}
\rule{1cm}{.4pt}, 2012b, {Supplementary material to
    ``Robust 3D gravity gradient inversion by planting anomalous densities''
    by Leonardo Uieda and Val\'eria C. F. Barbosa}: Figshare,
    http://hdl.handle.net/10779/cbfb817e4cc9ac3cf54bee7c649de1d3,
    accessed 25 April 2012.

\bibitem[Vasco, 1989]{vasco}
Vasco, D. W., 1989, Resolution and variance operators of gravity and gravity
    gradiometry: Geophysics, {\bf 54}, 889–899, doi: 10.1190/1.1442717.

\bibitem[Wilson et al., 2011]{wilson}
Wilson, G. A., M. Cuma, and M. S. Zhdanov, 2011, Large-scale 3D Inversion of
    Airborne Potential Field Data: 73rd EAGE Conference \& Exhibition
    incorporating SPE EUROPEC 2011, Expanded Abstracts, K047.

\bibitem[Zhdanov et al., 2004]{zhdanov04}
Zhdanov, M. S., R. G. Ellis, and S. Mukherjee, 2004, Regularized focusing
    inversion of 3-D gravity tensor data: Geophysics, {\bf 69},
    925--937, doi: 10.1190/1.1778236.

\bibitem[Zhdanov et al., 2010a]{zhdanov10a}
Zhdanov, M. S., A. Green, A. Gribenko, and M. Cuma, 2010a, Large-scale
    three-dimensional inversion of Earthscope MT data using the integral
    equation method: Physics of the Earth, {\bf 8}, 27--35,
    doi: 10.1134/S1069351310080045. 

\bibitem[Zhdanov et al., 2010b]{zhdanov10b}
Zhdanov, M. S., X. Liu, and G. A. Wilson, 2010b, Rapid imaging of gravity
    gradiometry data using 2D potential field migration: 80th Annual
    International Meeting, SEG, Expanded Abstracts, 1132--1136.
    
\bibitem[Zidarov and Zhelev, 1970]{zidarov}
Zidarov, D., and Z. Zhelev, 1970, On obtaining a family of bodies with identical
    exterior fields -- Method of bubbling: Geophysical Prospecting, {\bf 18},
    14--33.
\end{thebibliography}


%%%%%%%%%%%%%%%%%%%%%%%%%%%%%%%%%%%%%%%%%%%%%%%%%%%%%%%%%%%%%%%%%%%%%%%%%%%%%%%%
\newpage
\section{Figure Captions}


\begin{enumerate}

\item \label{fig:interpmodel}
    Schematic representation of the interpretative model
    consisting of a grid of $M$ juxtaposed 3D right-rectangular prisms.
    The interpretative model used to
    parametrize the anomalous density distribution
    is shown in gray.
    The observed $g_{yz}$ and $g_{zz}$ components of
    the gravity gradient tensor produced by
    the anomalous density distribution
    are shown in gray-scale contour maps.
    $\Delta x$, $\Delta y$, and $\Delta z$ are
    the lengths of the interpretative model
    in the $x$-, $y$-, and $z$-dimensions, respectively.

\item \label{fig:sketch}
    2D sketch of three stages of the planting algorithm.
    Black dots represent the observed data and the red line represents the
    predicted data produced by the current estimate.
    The light gray grid of prisms represents the interpretative model.
    (a) Initial state with the user-specified seeds included in the estimate
    with their corresponding density contrasts and all other parameters set to
    zero.
    (b) End of the first growth iteration where two accretions took place,
    one for each seed.
    The list of neighboring prisms of each seed and the predicted data are
    updated. (c) Final estimate at the end of the algorithm.
    The growth process stops when the predicted data fits the observed data.    
    
\item \label{fig:robust}
    2D sketch of the robust procedure.
    Black dots represent the observed data produced by (a) the true sources
    with different density contrasts $\rho_1$, $\rho_2$, and $\rho_3$
    (black, gray, and white polygons, respectively).
    The gray and white sources are considered
    non-targeted sources. The white source has a density contrast with the
    opposite sign of the black and gray sources.
    (b) Inversion result when given a seed only for the targeted source
    (black polygon) and using the $\ell_1$-norm of the residual vector
    (equation \ref{eq:phi}).
    The dashed line in b represents the data predicted by the inversion result.
    Large residuals over the non-targeted sources are automatically
    allowed by the inversion.
    The estimated density-contrast distribution (black prisms) recovers only
    the shape of the targeted source (black outline).

\item \label{fig:synthdata}
    Test with synthetic data produced by multiple
    targeted and non-targeted sources.
    (a) Synthetic noise-corrupted data (color-scale map) and data
    predicted by the inversion result (black contour lines) of the
    $g_{zz}$ component of the gravity
    gradient tensor.
    The synthetic data were produced by the 11 sources shown in Figure
    \ref{fig:synthresult}a.
    The predicted data is produced by the inversion result shown in Figure
    \ref{fig:synthresult}c.
    (b) The $g_{zz}$ component of the gravity gradient tensor
    produced only by the targeted sources
    (color-scale map) and
    the same data predicted by the inversion
    result in Figure \ref{fig:synthresult}c (black contour lines).

\item \label{fig:synthresult}
    Test with synthetic data produced by multiple
    targeted and non-targeted sources. (a) Perspective view of the synthetic
    model used to generate the synthetic data. Only sources with density
    contrast $0.6\ \mathrm{g/cm}^3$ (yellow) are outcropping.
    The sources with density contrast $1.2\ \mathrm{g/cm}^3$ (red) were
    considered as the targets of the interpretation.
    (b) Seeds used in the inversion and outline of the true targeted sources.
    (c) Inversion result obtained by using the $\ell_1$-norm of the residual
    vector (equation \ref{eq:phi}).
    Prisms of the interpretative model with zero density contrast are not
    shown. Black lines represent the outline of the true targeted sources.

\item \label{fig:fullseeds}
    Sensitivity analysis.
    Test using ideal seed locations and the correct density contrast of
    $1\ \mathrm{g/cm}^3$.
    The outline of the true sources is shown in solid black lines.
    The inversion result is shown as gray prisms.
    Prisms with zero density contrast are not shown.
    The inset shows the three seeds used in the inversion (gray prisms).
    The location of the seeds was chosen
    in order to outline the correct dip of
    the large dipping source (targeted source).
    
\item \label{fig:wrongseeds}
    Analysis of the sensitivity to uncertainties in the location of the seeds.
    Test using three seeds with the correct density contrast of
    $1\ \mathrm{g/cm}^3$ but with incorrect dip.
    The outline of the true sources is shown in solid black lines.
    The inversion result is shown as gray prisms.
    Prisms with zero density contrast are not shown.
    The inset shows the three seeds used in the inversion (gray prisms),
    which had the incorrect dip of
    the large dipping source (targeted source).

\item \label{fig:oneseed}
    Analysis of the sensitivity to the reduction of the number of seeds.
    Test using only a single seed and the correct density contrast of
    $1\ \mathrm{g/cm}^3$.
    The outline of the true sources is shown in solid black lines.
    The inversion result is shown as gray prisms.
    Prisms with zero density contrast are not shown.
    The inset shows the single seed used in the inversion (gray prism),
    which was located at the top of
    the large dipping source (targeted source).
    
\item \label{fig:samesign}
    Analysis of the limitations of the robust procedure.
    In this test, the smaller cubic source (non-targeted source) has
    a density contrast of $1.5\ \mathrm{g/cm}^3$, which has the same sign as the
    density contrast of the larger dipping source (targeted source).
    Test using three seeds in ideal locations
    (same as used in Figure \ref{fig:fullseeds})
    with the correct density contrast of
    $1\ \mathrm{g/cm}^3$.
    The inversion result is shown as gray prisms.
    Prisms with zero density contrast are not shown.
    The outline of the true sources is shown in solid black lines.
    The inset shows the three seeds used in the inversion (gray prisms).

\item \label{fig:realdata}
    Application to real data from an airborne gravity
    gradiometry survey over a region of the Quadril\'atero Ferr\'ifero,
    southeastern Brazil. The observed (a-c) and predicted (d-f) $g_{yy}$,
    $g_{yz}$, and $g_{zz}$ components of the gravity gradient tensor.
    The latter were produced by the estimated density-contrast distribution
    shown in Figure \ref{fig:realresult}.
    Black starts represent the horizontal coordinates of the seeds used in the
    inversion.

\item \label{fig:realresult} 
    Results from the application to real data from the
    Quadril\'atero Ferr\'ifero, southeastern Brazil.
    Dashed lines show the location of the cross-sections in Figure
    \ref{fig:realcross}.
    (a-c)
    Perspective views of the estimated density-contrast distribution,
    where prisms with zero density contrast are not shown
    or shown in gray and
    prisms with density contrast $1\ \mathrm{g/cm}^3$,
    corresponding to the iron orebody of the Cau\^{e} itabirite,
    are shown in solid or transparent red.
    The seeds used in the inversion are shown as black prisms.

\item \label{fig:realcross}
    Results from the application to real data from the
    Quadril\'atero Ferr\'ifero, southeastern Brazil.
    Cross-sections of the inversion result shown in Figure
    \ref{fig:realresult}
    at horizontal coordinate $x$ equal to (a) 1.00 km and (b) 5.55 km.
    Prisms with zero density contrast are shown in gray and
    prisms with density contrast $1\ \mathrm{g/cm}^3$,
    corresponding to the iron orebody of the Cau\^{e} itabirite,
    are shown in red. 
\end{enumerate}

\end{document}
